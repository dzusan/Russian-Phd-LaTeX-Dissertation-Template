\chapter*{Введение}							% Заголовок
\addcontentsline{toc}{chapter}{Введение}	% Добавляем его в оглавление

С каждым днём к нам приближается недалекое и так отчетливо видимое 
электронное будущее, которое принесет нам массу нововведений. Уже сегодня мы 
можем наблюдать за рождением новых, ярких идей и технологий. Одной из наиболее
интересных, перспективных и массовых технологий является идея создания 
беспилотного автотранспорта.

% В этой работе мы узнаем об основных причинах и целях создания и развития этой 
% технологии, что она обещает дать человечеству, какие негативные факторы может 
% устранить отсутствие человеческого фактора

На казалось бы сложный и объемный вопрос, с чего зародилась сама идея, история 
даёт достаточно простой ответ - все началось с тормозов. Первой <<атакой>> 
автомобильных конструкторов на водительские амбиции стало массовое применение 
антиблокировочной системы тормозов ABS. Создатели системы посчитали, что 
человек за рулем не способен справляться с блокировкой колес настолько 
эффективно, как это делает электроника. И, если первые подобные системы были 
несовершенны, то сейчас электроника намного эффективнее человека и уже никто 
не спорит о пользе ABS. Разобравшись с тормозами, конструкторы взялись за 
двигатель. Сначала на автомобилях появились антипробуксовочные системы, 
которые способны сдерживать мотор, если его мощность избыточна и приводит 
к пробуксовке ведущих колес. Затем появилась система стабилизации ESP, 
которой подчиняются не только двигатель, но и тормоза. В результате, 
ESP смогла самостоятельно бороться со сносами и заносами, выборочно 
подтормаживая колеса и регулируя тягу двигателя. Вскоре, разработчики 
электронных систем безопасности добрались до рулевого управления. 
% Оказалось, что "рулить" автоматика тоже может лучше человека. 
Например, система VDIM, 
способна доворачивать руль на несколько градусов, если того требует дорожная 
ситуация. Проще говоря, водитель-человек не способен провести автомобиль между 
конусами по идеальной траектории. Он не может воспроизвести один и тот же 
маневр с абсолютной точностью и т.д.

В наше время развитие беспилотного автотранспорта разделилось на 3 основных 
направления:

\begin{itemize}
    \item потребительское (личное авто, такси, городская авто транспортная сеть);
    \item промышленное (специализированная техника);
    \item военное (боевые машины различного спектра задач).
  \end{itemize}

В данный момент развитие беспилотного транспорта идет по всем перечисленным 
направлениям. Однако именно развитие потребительского беспилотного 
автотранспорта является основной задачей для общества. На мой взгляд это направление 
заслуживает особого внимания.

% 1. Описание базовой технологии

Развитие беспилотного автотранспорта для общества – должно быть приоритетной 
задачей для человечества, так как это позволит решить многие проблемы, в том 
числе значительно повысить безопасность и снизить смертность населения.
Дорожно-транспортный травматизм – одна из основных проблем общественного 
развития и здравоохранения. Более 3\% 
всех случаев смерти в мире -- это гибель в ДТП. Что сравнимо с числом смертей, 
вызванных такими массовыми заболеваниями как малярия и туберкулез. 

Однако развитие гражданского беспилотного транспорта встречает на своем пути 
множество проблем, таких как чрезмерно осторожное отношение самого общества,
моральная и законодательная сторона вопроса. Особого внимания требует проблема 
безопасности самого автопилота и ответственности при внедрении 
его на дороги общего пользования.

% Скопировано отсюда (с изменениями):
% https://domashke.net/referati/referaty-po-transportu/referat-bespilotnyj-avtotransport