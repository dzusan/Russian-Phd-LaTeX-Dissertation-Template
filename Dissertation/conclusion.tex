\chapter*{Заключение}						% Заголовок
\addcontentsline{toc}{chapter}{Заключение}	% Добавляем его в оглавление

% Pilotless_Integral
Число жертв автомобильных аварий в развитых странах стабильно уменьшается 
начиная с середины прошлого века, однако в данный момент можно наблюдать
некое плато: одна смерть на 100 
миллионов пройденных километров. Множество предпринятых мер, таких как подушки
безопасности и борьба с алкоголем за рулем сыграли свою роль. Необходимо 
совершать революционные действия по отношению к безопасности на дорогах.
На мой взгляд в этом как раз должен помочь беспилотный транспорт. Однако на пути его 
внедрения предстоит преодолеть ряд серьезных проблем, связанных в том числе и
с безопасностью самого по себе самопилотирования.

Пока в новостях люди слышат громкие заявления об авариях беспилотного транспорта.
Данную ситуацию можно сравнить с авиацией на заре 
ее развития, но теперь мы знаем, что самолет — это самый безопасный транспорт. 
Точно так же лет через 15—20 аварии беспилотных автомобилей будут являться
редкостью, трагическим стечением обстоятельств.
Беспилотные автомобили компании Гугл, уже 
несколько лет проходящие полевые испытания на дорогах Калифорнии, впервые попали 
в более-менее серьезную аварию с небольшими травмами пассажиров только после 2,7 
млн километров. С тех пор прошло уже больше года, эти автомобили прошли 
много сотен тысяч километров, а новых аварий больше не было. Я считаю, что в дальнейшем
ситуация должна только улучшаться.

% Plus&Minus
Мне хочется верить, что в скором времени человечество в корне изменит представление о передвижении.
Привычные для многих автомобили заменит более совершенный беспилотный транспорт. 
Большинство ученых уверены, что уже через 20 лет беспилотные автомобили
станут основным средством передвижения на дорогах общего пользования.
