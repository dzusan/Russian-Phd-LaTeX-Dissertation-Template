
{\actuality} С каждым днём к нам приближается недалекое и так отчетливо видимое 
электронное будущее, которое принесет нам массу нововведений. Уже сегодня мы 
можем наблюдать за рождением новых, ярких идей и технологий. Одной из наиболее
интересных, перспективных и массовых технологий является идея создания 
беспилотного автотранспорта.

% В этой работе мы узнаем об основных причинах и целях создания и развития этой 
% технологии, что она обещает дать человечеству, какие негативные факторы может 
% устранить отсутствие человеческого фактора

На казалось бы сложный и объемный вопрос, с чего зародилась сама идея, история 
даёт достаточно простой ответ - все началось с тормозов. Первой <<атакой>> 
автомобильных конструкторов на водительские амбиции стало массовое применение 
антиблокировочной системы тормозов ABS. Создатели системы посчитали, что 
человек за рулем не способен справляться с блокировкой колес настолько 
эффективно, как это делает электроника. И, если первые подобные системы были 
несовершенны, то сейчас электроника намного эффективнее человека и уже никто 
не спорит о пользе ABS. Разобравшись с тормозами, конструкторы взялись за 
двигатель. Сначала на автомобилях появились антипробуксовочные системы, 
которые способны сдерживать мотор, если его мощность избыточна и приводит 
к пробуксовке ведущих колес. Затем появилась система стабилизации ESP, 
которой подчиняются не только двигатель, но и тормоза. В результате, 
ESP смогла самостоятельно бороться со сносами и заносами, выборочно 
подтормаживая колеса и регулируя тягу двигателя. Вскоре, разработчики 
электронных систем безопасности добрались до рулевого управления. 
% Оказалось, что "рулить" автоматика тоже может лучше человека. 
Например, система VDIM, 
способна доворачивать руль на несколько градусов, если того требует дорожная 
ситуация. Проще говоря, водитель-человек не способен провести автомобиль между 
конусами по идеальной траектории. Он не может воспроизвести один и тот же 
маневр с абсолютной точностью и т.д.

В наше время развитие беспилотного автотранспорта разделилось на 3 основных 
направления:

\begin{itemize}
    \item потребительское (личное авто, такси, городская авто транспортная сеть);
    \item промышленное (специализированная техника);
    \item военное (боевые машины различного спектра задач).
  \end{itemize}

В данный момент развитие беспилотного транспорта идет по всем перечисленным 
направлениям. Однако именно развитие потребительского беспилотного 
автотранспорта является основной задачей для общества. Целью данной работы
поставлено выяснить, почему именно это направление заслуживает особого внимания.
Также рассмотреть проблему безопасности и ответственности при внедрении 
беспилотного транспорта на дороги общего пользования.

% 1. Описание базовой технологии

Развитие беспилотного автотранспорта для общества – должно быть приоритетной 
задачей для человечества. Создание Беспилотного автотранспорта в потребительской 
сфере:

\begin{itemize}
    \item исключит злоупотребление скоростью;
    \item исключит вождение в нетрезвом состоянии;
    \item поможет Службам неотложной помощи и поможет сократить объем и 
          количество пробок в мегаполисах.
  \end{itemize}

Дорожно-транспортный травматизм – одна из основных проблем общественного 
развития и здравоохранения. Ожидается, что масштаб этой проблемы в ближайшие 
годы значительно увеличится. Ежегодно около 1,2 миллиона человек во всем мире 
погибают в результате дорожно-транспортных аварий. Это составляет более 2,1\% 
всех случаев смерти в мире и сравнимо с числом смертей, вызванных такими 
массовыми заболеваниями малярией и туберкулезом. 
Еще больше число людей получают травмы и часто остаются инвалидами на всю жизнь.
Дорожно-транспортный травматизм – вторая лидирующая причина смерти среди лиц 
в возрасте 5–25 лет. В этой возрастной группе вероятность погибнуть или получить 
травму на дороге у молодых мужчин – пешеходов, велосипедистов, мотоциклистов, 
неопытных водителей и пассажиров – примерно в три раза выше, чем у молодых женщин.

% Скопировано отсюда:
% https://domashke.net/referati/referaty-po-transportu/referat-bespilotnyj-avtotransport